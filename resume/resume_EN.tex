\documentclass[10pt, letterpaper]{article}

% Packages:
\usepackage[
    ignoreheadfoot, % set margins without considering header and footer
    top=2 cm, % seperation between body and page edge from the top
    bottom=2 cm, % seperation between body and page edge from the bottom
    left=2 cm, % seperation between body and page edge from the left
    right=2 cm, % seperation between body and page edge from the right
    footskip=1.0 cm, % seperation between body and footer
    % showframe % for debugging 
]{geometry} % for adjusting page geometry
\usepackage{titlesec} % for customizing section titles
\usepackage{tabularx} % for making tables with fixed width columns
\usepackage{array} % tabularx requires this
\usepackage[dvipsnames]{xcolor} % for coloring text
\definecolor{primaryColor}{RGB}{0, 79, 144} % define primary color
\usepackage{enumitem} % for customizing lists
\usepackage{fontawesome5} % for using icons
\usepackage{amsmath} % for math
\usepackage[
    pdftitle={Marco Israel Rodríguez Cornejo CV},
    pdfauthor={Marco Rodríguez},
    pdfcreator={LaTeX with RenderCV},
    colorlinks=true,
    urlcolor=primaryColor
]{hyperref} % for links, metadata and bookmarks
\usepackage[pscoord]{eso-pic} % for floating text on the page
\usepackage{calc} % for calculating lengths
\usepackage{bookmark} % for bookmarks
\usepackage{lastpage} % for getting the total number of pages
\usepackage{changepage} % for one column entries (adjustwidth environment)
\usepackage{paracol} % for two and three column entries
\usepackage{ifthen} % for conditional statements
\usepackage{needspace} % for avoiding page brake right after the section title
\usepackage{iftex} % check if engine is pdflatex, xetex or luatex
\usepackage{etoolbox}
% Ensure that generate pdf is machine readable/ATS parsable:
\ifPDFTeX
    \input{glyphtounicode}
    \pdfgentounicode=1
    % \usepackage[T1]{fontenc} % this breaks sb2nov
    \usepackage[utf8]{inputenc}
    \usepackage{lmodern}
\fi



% Some settings:
\AtBeginEnvironment{adjustwidth}{\partopsep0pt} % remove space before adjustwidth environment
\pagestyle{empty} % no header or footer
\setcounter{secnumdepth}{0} % no section numbering
\setlength{\parindent}{0pt} % no indentation
\setlength{\topskip}{0pt} % no top skip
\setlength{\columnsep}{0cm} % set column seperation
\makeatletter
\let\ps@customFooterStyle\ps@plain % Copy the plain style to customFooterStyle
\patchcmd{\ps@customFooterStyle}{\thepage}{
    \color{gray}\textit{\small Marco Israel Rordríguez Cornejo - Página \thepage{} de \pageref*{LastPage}}
}{}{} % replace number by desired string
\makeatother
\pagestyle{customFooterStyle}

\titleformat{\section}{\needspace{4\baselineskip}\bfseries\large}{}{0pt}{}[\vspace{1pt}\titlerule]

\titlespacing{\section}{
    % left space:
    -1pt
}{
    % top space:
    0.3 cm
}{
    % bottom space:
    0.2 cm
} % section title spacing

\renewcommand\labelitemi{$\circ$} % custom bullet points
\newenvironment{highlights}{
    \begin{itemize}[
        topsep=0.10 cm,
        parsep=0.10 cm,
        partopsep=0pt,
        itemsep=0pt,
        leftmargin=0.4 cm + 10pt
    ]
}{
    \end{itemize}
} % new environment for highlights

\newenvironment{highlightsforbulletentries}{
    \begin{itemize}[
        topsep=0.10 cm,
        parsep=0.10 cm,
        partopsep=0pt,
        itemsep=0pt,
        leftmargin=10pt
    ]
}{
    \end{itemize}
} % new environment for highlights for bullet entries


\newenvironment{onecolentry}{
    \begin{adjustwidth}{
        0.2 cm + 0.00001 cm
    }{
        0.2 cm + 0.00001 cm
    }
}{
    \end{adjustwidth}
} % new environment for one column entries

\newenvironment{twocolentry}[2][]{
    \onecolentry
    \def\secondColumn{#2}
    \setcolumnwidth{\fill, 4.5 cm}
    \begin{paracol}{2}
}{
    \switchcolumn \raggedleft \secondColumn
    \end{paracol}
    \endonecolentry
} % new environment for two column entries

\newenvironment{header}{
    \setlength{\topsep}{0pt}\par\kern\topsep\centering\linespread{1.5}
}{
    \par\kern\topsep
} % new environment for the header

\newcommand{\placelastupdatedtext}{% \placetextbox{<horizontal pos>}{<vertical pos>}{<stuff>}
  \AddToShipoutPictureFG*{% Add <stuff> to current page foreground
    \put(
        \LenToUnit{\paperwidth-2 cm-0.2 cm+0.05cm},
        \LenToUnit{\paperheight-1.0 cm}
    ){\vtop{{\null}\makebox[0pt][c]{
        \small\color{gray}\textit{Julio 2025}\hspace{\widthof{Julio 2025}}
    }}}%
  }%
}%

% save the original href command in a new command:
\let\hrefWithoutArrow\href

% new command for external links:
\renewcommand{\href}[2]{\hrefWithoutArrow{#1}{\ifthenelse{\equal{#2}{}}{ }{#2 }\raisebox{.15ex}{\footnotesize \faExternalLink*}}}


\begin{document}
    \newcommand{\AND}{\unskip
        \cleaders\copy\ANDbox\hskip\wd\ANDbox
        \ignorespaces
    }
    \newsavebox\ANDbox
    \sbox\ANDbox{}

     %\placelastupdatedtext
    \begin{header}
        \textbf{\fontsize{24 pt}{24 pt}\selectfont Marco Israel Rodríguez Cornejo}

        \vspace{0.3 cm}

        \normalsize
        \mbox{{\color{black}\footnotesize\faMapMarker*}\hspace*{0.13cm}Ixtapaluca, México}%
        \kern 0.25 cm%
        \AND%
        \kern 0.25 cm%
        \mbox{\hrefWithoutArrow{mailto:marco\_israel@ciencias.unam.mx}{\color{black}{\footnotesize\faEnvelope[regular]}\hspace*{0.13cm}marco\_israel@ciencias.unam.mx}}%
        \kern 0.25 cm%
        \AND%
        \kern 0.25 cm%
        \mbox{\hrefWithoutArrow{tel:+525581501458}{\color{black}{\footnotesize\faPhone*}\hspace*{0.13cm}52 55 8150 14 58}}%
        \kern 0.25 cm%
        \AND%
        \kern 0.25 cm%
        \mbox{\hrefWithoutArrow{https://marcoisrael.github.io/}{\color{black}{\footnotesize\faLink}\hspace*{0.13cm}marcoisrael.github.io}}%
        \kern 0.25 cm%
        \AND%
        \kern 0.25 cm%
        \mbox{\hrefWithoutArrow{https://linkedin.com/in/marco-israel-rodriguez-cornejo-b6949a276}{\color{black}{\footnotesize\faLinkedinIn}\hspace*{0.13cm}marco-israel-rodriguez-cornejo-b6949a276}}%
        \kern 0.25 cm%
        \AND%
        \kern 0.25 cm%
        \mbox{\hrefWithoutArrow{https://github.com/marcoisrael}{\color{black}{\footnotesize\faGithub}\hspace*{0.13cm}marcoisrael}}%
    \end{header}

    \vspace{0.3 cm - 0.3 cm}


    \section{About}



        
        \begin{onecolentry}
            I am a physicist focused on data science, numerical simulations, and machine learning. During my training, I conducted simulations of complex physical phenomena using various programming languages and tools such as Python and Fortran at the Institute of Nuclear Sciences' high-performance data center.       \end{onecolentry}

        \vspace{0.2 cm}

    \section{Education}



        
        \begin{twocolentry}{
                \textit{2017 - 2022}}
                \textbf{Facultad de Ciencias UNAM}
                
                \textit{Bachelor's degree in physics}            
        \end{twocolentry}
        \vspace{0.2cm}
        \begin{twocolentry}{
            \textit{2022 - 2023}}
                \textbf{Instituto de Ciencias Nucleares UNAM}

                \textit{Intern}
        \end{twocolentry}
        \vspace{0.2cm}
        \begin{twocolentry}{
                \textit{2023 - 2024}}
                    \textbf{Tecnológico Nacional de México}
                    
                    \textit{Data science certificate}
        \end{twocolentry}


    
    \section{Experience}


        
        \begin{twocolentry}{
            
            \textit{March 2023 - July 2025}}
            \textbf{Instituto de Ciencias Nucleares UNAM}
            
            \href{https://raw.githubusercontent.com/marcoisrael/SigmaModel/master/Marco_Tesis.pdf}{Enfriamientos rápidos del modelo 2d O(3)}
        \end{twocolentry}

        \vspace{0.10 cm}
        \begin{onecolentry}
           My thesis project is a study of rapid cooling in a physical model called 2d O(3). The research required the application of Monte Carlo methods, Markov chains, and advanced statistics, as well as the development of an efficient program capable of performing the simulations, guided by Dr. Wolfgang Peter Bietenholz.
        \end{onecolentry}


        \vspace{0.2 cm}

        \begin{twocolentry}{
            
            \textit{August 2017}}
            \textbf{Latin American Journal of Science Education}
            
            \href{https://www.lajse.org/nov17/22023_Gonzalez_2017.pdf}{Amplitudes grandes de un péndulo simple amortiguado}
            
        \end{twocolentry}

        \vspace{0.10 cm}
        \begin{onecolentry}
I collaborated on an article on numerical simulations in a physical system for the Latin American Journal of Science Education with the advice of Dr. Alejandro González y Hernández. I conducted a controlled experiment in the laboratory and wrote a program to solve the problem using the second-order Euler method.
        \end{onecolentry}

    
    
    \section{Aptitudes}



        
        \begin{onecolentry}
            \textbf{Programming languages:} Python, Java, Fortran, R
        \end{onecolentry}

        \vspace{0.2 cm}

        \begin{onecolentry}
            \textbf{Tools:} SQL, Numpy, Matplotlib, Pandas, Tensorflow, Scikit-learn
        \end{onecolentry}

        \vspace{0.2cm}

        \begin{onecolentry}
            \textbf{Languages:} Spanish, English
        \end{onecolentry}


    

\end{document}

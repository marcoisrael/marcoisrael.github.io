\documentclass[10pt, letterpaper]{article}

% Packages:
\usepackage[
    ignoreheadfoot, % set margins without considering header and footer
    top=2 cm, % seperation between body and page edge from the top
    bottom=2 cm, % seperation between body and page edge from the bottom
    left=2 cm, % seperation between body and page edge from the left
    right=2 cm, % seperation between body and page edge from the right
    footskip=1.0 cm, % seperation between body and footer
    % showframe % for debugging 
]{geometry} % for adjusting page geometry
\usepackage{titlesec} % for customizing section titles
\usepackage{tabularx} % for making tables with fixed width columns
\usepackage{array} % tabularx requires this
\usepackage[dvipsnames]{xcolor} % for coloring text
\definecolor{primaryColor}{RGB}{0, 79, 144} % define primary color
\usepackage{enumitem} % for customizing lists
\usepackage{fontawesome5} % for using icons
\usepackage{amsmath} % for math
\usepackage[
    pdftitle={Marco Israel Rodríguez Cornejo CV},
    pdfauthor={Marco Rodríguez},
    pdfcreator={LaTeX with RenderCV},
    colorlinks=true,
    urlcolor=primaryColor
]{hyperref} % for links, metadata and bookmarks
\usepackage[pscoord]{eso-pic} % for floating text on the page
\usepackage{calc} % for calculating lengths
\usepackage{bookmark} % for bookmarks
\usepackage{lastpage} % for getting the total number of pages
\usepackage{changepage} % for one column entries (adjustwidth environment)
\usepackage{paracol} % for two and three column entries
\usepackage{ifthen} % for conditional statements
\usepackage{needspace} % for avoiding page brake right after the section title
\usepackage{iftex} % check if engine is pdflatex, xetex or luatex
\usepackage{etoolbox}
% Ensure that generate pdf is machine readable/ATS parsable:
\ifPDFTeX
    \input{glyphtounicode}
    \pdfgentounicode=1
    % \usepackage[T1]{fontenc} % this breaks sb2nov
    \usepackage[utf8]{inputenc}
    \usepackage{lmodern}
\fi



% Some settings:
\AtBeginEnvironment{adjustwidth}{\partopsep0pt} % remove space before adjustwidth environment
\pagestyle{empty} % no header or footer
\setcounter{secnumdepth}{0} % no section numbering
\setlength{\parindent}{0pt} % no indentation
\setlength{\topskip}{0pt} % no top skip
\setlength{\columnsep}{0cm} % set column seperation
\makeatletter
\let\ps@customFooterStyle\ps@plain % Copy the plain style to customFooterStyle
\patchcmd{\ps@customFooterStyle}{\thepage}{
    \color{gray}\textit{\small Marco Israel Rordríguez Cornejo - Página \thepage{} de \pageref*{LastPage}}
}{}{} % replace number by desired string
\makeatother
\pagestyle{customFooterStyle}

\titleformat{\section}{\needspace{4\baselineskip}\bfseries\large}{}{0pt}{}[\vspace{1pt}\titlerule]

\titlespacing{\section}{
    % left space:
    -1pt
}{
    % top space:
    0.3 cm
}{
    % bottom space:
    0.2 cm
} % section title spacing

\renewcommand\labelitemi{$\circ$} % custom bullet points
\newenvironment{highlights}{
    \begin{itemize}[
        topsep=0.10 cm,
        parsep=0.10 cm,
        partopsep=0pt,
        itemsep=0pt,
        leftmargin=0.4 cm + 10pt
    ]
}{
    \end{itemize}
} % new environment for highlights

\newenvironment{highlightsforbulletentries}{
    \begin{itemize}[
        topsep=0.10 cm,
        parsep=0.10 cm,
        partopsep=0pt,
        itemsep=0pt,
        leftmargin=10pt
    ]
}{
    \end{itemize}
} % new environment for highlights for bullet entries


\newenvironment{onecolentry}{
    \begin{adjustwidth}{
        0.2 cm + 0.00001 cm
    }{
        0.2 cm + 0.00001 cm
    }
}{
    \end{adjustwidth}
} % new environment for one column entries

\newenvironment{twocolentry}[2][]{
    \onecolentry
    \def\secondColumn{#2}
    \setcolumnwidth{\fill, 4.5 cm}
    \begin{paracol}{2}
}{
    \switchcolumn \raggedleft \secondColumn
    \end{paracol}
    \endonecolentry
} % new environment for two column entries

\newenvironment{header}{
    \setlength{\topsep}{0pt}\par\kern\topsep\centering\linespread{1.5}
}{
    \par\kern\topsep
} % new environment for the header

\newcommand{\placelastupdatedtext}{% \placetextbox{<horizontal pos>}{<vertical pos>}{<stuff>}
  \AddToShipoutPictureFG*{% Add <stuff> to current page foreground
    \put(
        \LenToUnit{\paperwidth-2 cm-0.2 cm+0.05cm},
        \LenToUnit{\paperheight-1.0 cm}
    ){\vtop{{\null}\makebox[0pt][c]{
        \small\color{gray}\textit{Julio 2025}\hspace{\widthof{Julio 2025}}
    }}}%
  }%
}%

% save the original href command in a new command:
\let\hrefWithoutArrow\href

% new command for external links:
\renewcommand{\href}[2]{\hrefWithoutArrow{#1}{\ifthenelse{\equal{#2}{}}{ }{#2 }\raisebox{.15ex}{\footnotesize \faExternalLink*}}}


\begin{document}
    \newcommand{\AND}{\unskip
        \cleaders\copy\ANDbox\hskip\wd\ANDbox
        \ignorespaces
    }
    \newsavebox\ANDbox
    \sbox\ANDbox{}

     %\placelastupdatedtext
    \begin{header}
        \textbf{\fontsize{24 pt}{24 pt}\selectfont Marco Israel Rodríguez Cornejo}

        \vspace{0.3 cm}

        \normalsize
        \mbox{{\color{black}\footnotesize\faMapMarker*}\hspace*{0.13cm}Ixtapaluca, México}%
        \kern 0.25 cm%
        \AND%
        \kern 0.25 cm%
        \mbox{\hrefWithoutArrow{mailto:marco\_israel@ciencias.unam.mx}{\color{black}{\footnotesize\faEnvelope[regular]}\hspace*{0.13cm}marco\_israel@ciencias.unam.mx}}%
        \kern 0.25 cm%
        \AND%
        \kern 0.25 cm%
        \mbox{\hrefWithoutArrow{tel:+525581501458}{\color{black}{\footnotesize\faPhone*}\hspace*{0.13cm}52 55 8150 14 58}}%
        \kern 0.25 cm%
        \AND%
        \kern 0.25 cm%
        \mbox{\hrefWithoutArrow{https://marcoisrael.github.io/}{\color{black}{\footnotesize\faLink}\hspace*{0.13cm}marcoisrael.github.io}}%
        \kern 0.25 cm%
        \AND%
        \kern 0.25 cm%
        \mbox{\hrefWithoutArrow{https://linkedin.com/in/marco-israel-rodriguez-cornejo-b6949a276}{\color{black}{\footnotesize\faLinkedinIn}\hspace*{0.13cm}marco-israel-rodriguez-cornejo-b6949a276}}%
        \kern 0.25 cm%
        \AND%
        \kern 0.25 cm%
        \mbox{\hrefWithoutArrow{https://github.com/marcoisrael}{\color{black}{\footnotesize\faGithub}\hspace*{0.13cm}marcoisrael}}%
    \end{header}

    \vspace{0.3 cm - 0.3 cm}


    \section{Sobre mi}



        
        \begin{onecolentry}
            Soy un físico enfocado en la ciencia de datos, simulaciones numéricas y aprendizaje automático. Durante mi formación realicé simulaciones de fenómenos físicos complejos empleando diversos lenguajes de programación y herramientas como Python y Fortran en el centro de datos de alto rendimiento del Instituto de Ciencias Nucleares. 
        \end{onecolentry}

        \vspace{0.2 cm}

    \section{Educación}



        
        \begin{twocolentry}{
                \textit{2017 - 2022}}
                \textbf{Facultad de Ciencias UNAM}
                
                \textit{Licenciatura en Física}            
        \end{twocolentry}
        \vspace{0.2cm}
        \begin{twocolentry}{
            \textit{2022 - 2023}}
                \textbf{Instituto de Ciencias Nucleares UNAM}

                \textit{Pasantía}
        \end{twocolentry}


    
    \section{Experiencia}


        
        \begin{twocolentry}{
            
            \textit{Mazo 2023 - Julio 2025}}
            \textbf{Instituto de Ciencias Nucleares UNAM}
            
            \href{https://raw.githubusercontent.com/marcoisrael/SigmaModel/master/Marco_Tesis.pdf}{Enfriamientos rápidos del modelo 2d O(3)}
        \end{twocolentry}

        \vspace{0.10 cm}
        \begin{onecolentry}
            Mi proyecto de titulación, es un estudio sobre enfriamientos rápidos en un modelo físico llamado 2d O(3). La investigación requirió de la aplicación de métodos de Montecarlo, cadenas de Markov y estadística avanzada, así como el desarrollo de un programa eficiente capaz de realizar las simulaciones con guía del Dr. Wolfgang Peter Bietenholz.
        \end{onecolentry}


        \vspace{0.2 cm}

        \begin{twocolentry}{
            
            \textit{Agosto 2017}}
            \textbf{Latin American Journal of Science Education}
            
            \href{https://www.lajse.org/nov17/22023_Gonzalez_2017.pdf}{Amplitudes grandes de un péndulo simple amortiguado}
            
        \end{twocolentry}

        \vspace{0.10 cm}
        \begin{onecolentry}
            Colabore en un artículo sobre simulaciones numéricas en un sistema físico para la revista Latin American Journal of Science Education con la asesoría del Dr. Alejandro González y Hernández. Realice un experimento controlado en el laboratorio y se escribi un programa para resolver el problema usando el método de euler de segundo orden.
        \end{onecolentry}

    
    
    \section{Aptitudes}



        
        \begin{onecolentry}
            \textbf{Lenguajes de programación:} Python, Java, Fortran
        \end{onecolentry}

        \vspace{0.2 cm}

        \begin{onecolentry}
            \textbf{Herrameintas:} SQL, Numpy, Matplotliib, Pandas, Tensorflow
        \end{onecolentry}

        \vspace{0.2cm}

        \begin{onecolentry}
            \textbf{Idiomas:} Inglés, Español
        \end{onecolentry}


    

\end{document}
